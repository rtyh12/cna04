\documentclass {article}
\usepackage{amsmath}
\setlength{\parindent}{0cm}

\usepackage{hyperref}
\usepackage{graphicx}
\usepackage{subcaption}
\usepackage[margin=1.5in]{geometry}

\usepackage{fancyhdr}
\pagestyle{fancy}
\lhead{\textbf{Complex Network Analysis} \\ Assignment 3\\}
\rhead{Maria Kagkeli \\ Maria Regina Lily \\ Mihai Verzan}
\headheight 10pc
\voffset -10pc

\begin{document}



% problem 1
\section{Power Laws}

\subsection{}
(a) $ \to $ not scale-free, because the plot is curved, therefore $ \log p_k $ does not depend linearly on $ \log k $.

(b) $ \to $ scale-free, because $ \log p_k $ depends linearly on $ \log k $ (the plot is approximately a straight line).

\subsection{}
$ \log_{10} p_k \sim - \gamma \cdot log_{10} k \Rightarrow
  \gamma = - \frac{ \log_{10} p_k }{ \log_{10} k } $. 
Sample several points on the graph and estimate the values in those points, then plug them in the formula:

\begin{itemize}
  \item $ k = 10 $, $ p_k = 10^{-2} \Rightarrow \gamma = 2 $
  \item $ k = 2  $, $ p_k = 10^{-1} \Rightarrow \gamma = 3.321 $
  \item $ k = 50 $, $ p_k = 10^{-3} \Rightarrow \gamma = 1.765 $
\end{itemize}

We can estimate $ \gamma $ to be around 2.

\subsection{}
$ \gamma = 1 + N[\sum^N_{i=1} \ln \frac{ K_i }{ K_{min}-\frac{1}{2} }] = 1.756 $, $ \sigma = 0.16913 $ (values calculated using a Python script. See 4-1.ipynb).

\newpage

\end{document}